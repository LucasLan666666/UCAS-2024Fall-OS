\question {
    现有一块磁盘,假设其磁头当前位于第 103 磁道,正在向磁道序号增加的方向移动。现有一个磁盘
    访问请求序列,其访问的磁道号依次为33,50,8,69,110,150,173,202,请计算:
}

\begin{parts}
    \part {
        当分别采用 FIFO、SSF 和 C-SCAN 三种磁盘调度算法执行上述磁盘请求序列时,三种情况
        下的寻道距离各是多少?
    }
\end{parts}

\begin{solution}

对于 FIFO 算法,磁头按照请求的顺序进行访问,所以服务顺序是:
$$
    103 \rightarrow 33 \rightarrow 50 \rightarrow 8 \rightarrow 69
    \rightarrow 110 \rightarrow 150 \rightarrow 173 \rightarrow 202
$$
因此寻道距离是:
$$
\begin{aligned}
    D_{\text{FIFO}} &= |103 - 33| + |33 - 50| + |50 - 8| + |8 - 69| + |69 - 110|
                       + |110 - 150| + |150 - 173| + |173 - 202| \\
                    &= 70 + 17 + 42 + 61 + 41 + 40 + 23 + 29 \\
                    &= 323
\end{aligned}
$$

对于 SSF 算法,磁头按照离当前磁道最近的磁道进行访问,所以服务顺序是:
$$
    103 \rightarrow 110 \rightarrow 150 \rightarrow 173 \rightarrow 202
    \rightarrow 69 \rightarrow 50 \rightarrow 33 \rightarrow 8
$$
因此寻道距离是:
$$
\begin{aligned}
    D_{\text{SSF}} &= |103 - 110| + |110 - 150| + |150 - 173| + |173 - 202|
                     + |202 - 69| + |69 - 50| + |50 - 33| + |33 - 8| \\
                  &= 7 + 40 + 23 + 29 + 133 + 19 + 17 + 25 \\
                  &= 293
\end{aligned}
$$

对于 C-SCAN 算法,假定磁头向磁道序号增加的方向移动,那么服务顺序是:
$$
    103 \rightarrow 110 \rightarrow 150 \rightarrow 173 \rightarrow 202
    \rightarrow 8 \rightarrow 33 \rightarrow 50 \rightarrow 69
$$
因此寻道距离是:
$$
\begin{aligned}
    D_{\text{C-SCAN}} &= |103 - 110| + |110 - 150| + |150 - 173| + |173 - 202|
                        + |202 - 8| + |8 - 33| + |33 - 50| + |50 - 69| \\
                     &= 7 + 40 + 23 + 29 + 194 + 25 + 17 + 19 \\
                     &= 354
\end{aligned}
$$

\end{solution}
