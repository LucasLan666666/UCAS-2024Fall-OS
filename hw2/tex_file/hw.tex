\section{作业}

\subsection{
    一个C程序可以编译成目标文件或可执行文件。目标文件和可执行文件通常包含{\tt text}、{\tt data}、{\tt bss}、{\tt rodata}段,程序执行时也会用到堆({\tt heap})和栈({\tt stack})
}

\subsubsection{
    请写一个C程序,使其包含{\tt data}段和{\tt bss}段,并在运行时包含堆的使用。请说明所写程序中哪些变量在{\tt data}段、{\tt bss}段和堆上
}

\noindent
我写的程序如下:

\lstinputlisting[language=C, caption={prog1.c}]{code/prog1.c}

\noindent
变量的位置如下:

\begin{tabular}{|c|c|c|}
    \hline
    变量 & 所在位置 & 原因 \\
    \hline
    {\tt a} & {\tt data}段 & 初始化的全局变量 \\
    \hline
    {\tt b} & {\tt bss}段 & 未初始化的全局变量 \\
    \hline
    {\tt c} & {\tt data}段 & 静态变量 \\
    \hline
    {\tt *p} & 堆上 & 通过{\tt malloc}函数分配得到 \\
    \hline
\end{tabular}

\subsubsection{
    请了解{\tt readelf}、{\tt objdump}命令的使用,用这些命令查看(1)中所写程序的{\tt data}和{\tt bss}段,截图展示
}

\noindent
如下图所示,可以看到{\tt data}段和{\tt bss}段的内容:

\begin{figure}[H]
    \centering
    \includegraphics[width=1\textwidth]{fig/data_bss.png}
    \caption{{\tt data}和{\tt bss}段}
\end{figure}

\noindent
{\tt data}段的$0$和$1$刚好对应{\tt a}和{\tt c}的值,{\tt bss}段对应{\tt b}的值,符合预期。

\subsubsection{
    请说明(1)中所写程序是否用到了栈
}

\noindent
用到了栈,理由如下:

一方面,{\tt p}是一个局部变量,它的地址是在栈上分配的;另一方面,进入函数{\tt malloc()}的时候需要将返回地址压栈。

\subsection{
    {\tt fork}、{\tt exec}、{\tt wait}等是进程操作的常用API,请调研了解这些API的使用方法
}

\subsubsection{
    请写一个{\tt C}程序,该程序首先创建一个$1$到$10$的整数数组,然后创建一个子进程,并让子进程对前述数组所有元素求和,并打印求和结果。等子进程完成求和后,父进程打印{\tt "parentprocess finishes"},再退出
}

\noindent
我写的程序如下:

\lstinputlisting[language=C, caption={prog2.c}]{code/prog2.c}

\subsubsection{
    在(1)所写的程序基础上,当子进程完成数组求和后,让其执行{\tt ls -l}命令(注:该命令用于显示某个目录下文件和子目录的详细信息),显示你运行程序所用操作系统的某个目录详情。例如,让子进程执行{\tt ls -l /usr/bin}目录,显示{\tt /usr/bin}目录下的详情。父进程仍然需要等待子进程执行完后打印{\tt "parentprocess finishes"},再退出
}

\noindent
修改后的程序如下:

\lstinputlisting[language=C, caption={prog3.c}]{code/prog3.c}

\noindent
运行结果如下:

\begin{figure}[H]
    \centering
    \includegraphics[width=1\textwidth]{fig/ls.png}
    \caption{运行结果}
\end{figure}

\subsubsection{
    请阅读XV6代码(https://pdos.csail.mit.edu/6.828/2024/xv6.html),找出XV6代码中对进程控制块({\tt PCB})的定义代码,说明其所在的文件,以及当{\tt fork}执行时,对{\tt PCB}做了哪些操作?
}

\noindent
{\tt PCB}的定义在{\tt kernel/proc.h}文件中,如下所示:

\begin{lstlisting}[language=C, caption={{\tt PCB}定义}]

    // Per-process state
    struct proc {
      struct spinlock lock;

      // p->lock must be held when using these:
      enum procstate state;        // Process state
      void *chan;                  // If non-zero, sleeping on chan
      int killed;                  // If non-zero, have been killed
      int xstate;                  // Exit status to be returned to parent's wait
      int pid;                     // Process ID

      // wait_lock must be held when using this:
      struct proc *parent;         // Parent process

      // these are private to the process, so p->lock need not be held.
      uint64 kstack;               // Virtual address of kernel stack
      uint64 sz;                   // Size of process memory (bytes)
      pagetable_t pagetable;       // User page table
      struct trapframe *trapframe; // data page for trampoline.S
      struct context context;      // swtch() here to run process
      struct file *ofile[NOFILE];  // Open files
      struct inode *cwd;           // Current directory
      char name[16];               // Process name (debugging)
    };

\end{lstlisting}

\noindent
当{\tt fork}执行时,对{\tt PCB}做了如下操作:

\begin{enumerate}
    \item 首先,定义变量{\tt p}和{\tt np},分别是指向父进程和子进程结构体的指针;
    \item 通过{\tt allocproc()}函数分配一个新的进程结构体,即子进程;
    \item 调用{\tt uvmcopy()}函数将父进程的内存信息复制给子进程;
    \item 将父进程的用户寄存器复制给子进程;
    \item 将子进程的返回值设置为{\tt 0};
    \item 用{\tt filedup()}函数将父进程的文件描述符复制给子进程;
    \item 用{\tt idup}函数将父进程的工作目录复制给子进程;
    \item 用{\tt safestrcpy()}复制进程名称;
    \item 释放子进程锁,设置子进程的父进程,设置子进程状态为可运行;
    \item 最后返回子进程的{\tt pid}。
\end{enumerate}

\subsection{
    请阅读以下程序代码,回答下列问题
}

\lstinputlisting[language=C, caption={程序代码}]{code/example.c}

\subsubsection{
    该程序一共会生成几个子进程?请你画出生成的进程之间的关系(即谁是父进程谁是子进程),并对进程关系进行适当说明
}

\noindent
该程序一共生成了$3$个子进程。进程关系如下:

\begin{figure}[H]
    \centering
    \includegraphics[width=1\textwidth]{fig/fork.png}
    \caption{进程关系}
\end{figure}

\noindent
说明:

\begin{enumerate}
    \item 设初始进程为{\tt p0},{\tt p0}在进行第一次{\tt fork()}系统调用后创造了子进程{\tt p1}。
    \item {\tt p0}休眠$5$秒;{\tt p1}在终端打印信息,{\tt loop}增加到$1$,同时立刻进行{\tt fork()}系统调用,创造子进程{\tt p2}。
    \item {\tt p2}在终端打印信息,{\tt loop}增加到$2$,结束循环,终止。
    \item {\tt p0}第一次休眠结束,{\tt loop}增加到$1$,进行{\tt fork()}系统调用后创建子进程{\tt p3},同时开始第二次休眠;{\tt p3}在终端打印信息后{\tt loop}增加到$2$,结束循环,终止。
    \item {\tt p1}休眠结束,{\tt loop}增加到$2$,结束循环,终止。
    \item {\tt p0}第二次休眠结束,{\tt loop}增加到$2$,结束循环,终止。
\end{enumerate}

\subsubsection{
    如果生成的子进程数量和宏定义{\tt LOOP}不符,在不改变{\tt for}循环的前提下,你能用少量代码修改,使该程序生成{\tt LOOP}个子进程么?
}

\noindent
有,我的方法是让父进程在创建完子进程后立刻终止,修改后代码如下:

\lstinputlisting[language=C, caption={程序代码}]{code/example_mod.c}
