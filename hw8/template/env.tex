\lhead{\Homework \\}
\rhead{\today \\}
% \chead{\hline} % 中部横线
\thispagestyle{empty} % 首页不显示页码


% 偏好的几个字体, 可以根据需要自行加入字体ttf文件并调用
% 注意要在自己系统安装字体, 以供调用
% 检查并设置中文主字体
\IfFontExistsTF{Source Han Serif CN}
  {\setCJKmainfont{Source Han Serif CN}}
  {\IfFontExistsTF{Noto Serif CJK SC}
    {\setCJKmainfont{Noto Serif CJK SC}}
    {\setCJKmainfont{SimSun}}} % Default font
% 检查并设置等宽字体
\IfFontExistsTF{JetBrains Mono NL}
  {\setmonofont{JetBrains Mono NL}}
  {\setmonofont{Courier New}} % 默认等宽字体
% 检查并设置英文主字体
\punctstyle{kaiming}

% 定义中文环境
\renewcommand \thesection{\zhnum{section}、}{\hangindent=2em\hangafter=1}
\renewcommand \thesubsection{\arabic{subsection}}{\hangindent=2em\hangafter=1}
\renewcommand \thesubsubsection{(\arabic{subsubsection})}{\hangindent=2em\hangafter=1}
\setlength{\parindent}{0pt} % 取消首行缩进

% 代码块设置
\definecolor{bluekeywords}{rgb}{0.13, 0.13, 1}
\definecolor{greencomments}{rgb}{0, 0.5, 0}
\definecolor{redstrings}{rgb}{0.9, 0, 0}
\definecolor{graynumbers}{rgb}{0.5, 0.5, 0.5}
\lstset{
    autogobble, % 自动移除代码前的空白(对齐代码块)
    columns=fixed, % 使得空格不可拉伸
    showspaces=false, % 不显示空格
    showtabs=false, % 不显示制表符
    breaklines=true, % 自动换行
    showstringspaces=false, % 在字符串中不特别显示空格
    breakatwhitespace=true, % 仅在空白处进行自动换行
    escapeinside={(*@}{@*)}, % 设置一个特殊的标记,允许在代码中插入LaTeX代码
    commentstyle=\color{greencomments}, % 注释的颜色设置为绿色
    keywordstyle=\color{bluekeywords}, % 关键字的颜色设置为蓝色
    stringstyle=\color{redstrings}, % 字符串的颜色设置为红色
    numberstyle=\color{graynumbers}, % 行号的颜色设置为灰色
    identifierstyle=\color{orange}, % 标识符颜色
    basicstyle=\small\ttfamily, % 基本字体样式:小号的等宽字体
    frame=single, % 单线边框
    framesep=5pt, % 边框与代码的间隔
    framerule=0.75pt, % 边框宽度
    xleftmargin=0pt, % 左边距
    xrightmargin=0pt, % 右边距
    tabsize=4, % 制表符宽度
    captionpos=t, % 标题位置在顶部
}
\DeclareCaptionFont{white}{\color{white}}
\DeclareCaptionFormat{listing}{\colorbox[cmyk]{0.43, 0.35, 0.35,0.01}
{\parbox{\textwidth}{\hspace{15pt}#1#2#3}}}
\captionsetup[lstlisting]{format=listing,labelfont=white,textfont=white,
                          singlelinecheck=false, margin=0pt,
                          font={bf,footnotesize}}

\renewcommand{\solutiontitle}{\noindent\textbf{解答:}\\} % 答案前缀
\renewcommand{\arraystretch}{1.5} % 表格行高
