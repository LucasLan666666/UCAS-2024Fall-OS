\question {
    在生产者——消费者问题中,假设缓冲区大小为$5$,生产者和消费者在写入和读取数据时都会更新写入/读取的位置{\tt offset}。现有以下两种基于信号量的实现方法
}

方法一:

\begin{lstlisting}[language=C++]
Class BoundedBuffer {
    mutex = new Semaphore(1);
    fullBuffers = new Semaphore(0);
    emptyBuffers = new Semaphore(n);
}
BoundedBuffer::Deposit(c) {
    emptyBuffers->P();
    mutex->P();
    Add c to the buffer;
    offset++;
    mutex->V();
    fullBuffers->V();
}
BoundedBuffer::Remove(c) {
    fullBuffers->P();
    mutex->P();
    Remove c from buffer;
    offset--;
    mutex->V();
    emptyBuffers->V();
}
\end{lstlisting}

方法二:

\begin{lstlisting}[language=C++]
Class BoundedBuffer {
    mutex = new Semaphore(1);
    fullBuffers = new Semaphore(0);
    emptyBuffers = new Semaphore(n);
}
BoundedBuffer::Deposit(c) {
    mutex->P();
    emptyBuffers->P();
    Add c to the buffer;
    offset++;
    fullBuffers->V();
    mutex->V();
}
BoundedBuffer::Remove(c) {
    mutex->P();
    fullBuffers->P();
    Remove c from buffer;
    offset--;
    emptyBuffers->V();
    mutex->V();
}
\end{lstlisting}

请对比分析上述方法一和方法二,哪种方法能让生产者、消费者进程正常运行,并说明分析原因。

\begin{solution}

方法一能让生产者、消费者进程正常运行,而方法二不能。理由如下:

方法一和方法二的唯一区别就是在对{\tt BoundedBuffer}进行{\tt Deposit()}和{\tt Remove()}操作时,获取和释放互斥锁的位置不同,换言之,临界区不一样。

在方法二中,生产者和消费者操作的互斥锁包含了整个操作过程。假如生产者在进行{\tt Deposit()}操作时并获取了互斥锁之后,缓冲区已满,那么生产者会被阻塞在{\tt emptyBuffers->P()}处,进而无法完成对互斥锁的释放,导致消费者也无法进行{\tt Remove()}操作,从而产生死锁。而方法一中,生产者和消费者操作的互斥锁只包含了对缓冲区的操作,因此生产者在进行{\tt Deposit()}操作时,即使被阻塞在{\tt emptyBuffers->P()}处,也不会影响消费者进行{\tt Remove()}操作,从而避免了死锁的发生。

如果消费者在进行{\tt Remove()}操作时发现缓冲区为空,也会出现类似问题,不再赘述。

\end{solution}
