\question {
    两进程$A$和$B$各需要数据库中的$3$份记录$1$、$2$、$3$,若进程$A$以$1$、$2$、$3$的顺序请求这些资源, 进程$B$也以同样的顺序请求这些资源,则将不会产生死锁。但若进程$B$以$3$、$2$、$1$的顺序请求这些资源,则可能会产生死锁。这$3$份资源存在$6$种可能的请求顺序,其中哪些请求顺序能保证无死锁产生?请写出解题分析过程。
}

\begin{solution}

只有$1 \to 2 \to 3$和$1 \to 3 \to 2$两种请求顺序可能产生死锁。理由如下:

\begin{itemize}
    \item 对于$1 \to 2 \to 3$ 和 $1 \to 3 \to 2$两种请求顺序,$A$和$B$必定有且只有一个进程能够顺利请求到$1$资源。此时另一个进程必定因为首先请求$1$失败而被阻塞,直到前一个进程请求完所有资源后,才会释放$1$,后一个进程才能成功请求到$1$,因此能顺利请求所有资源,保证不会产生死锁。
    \item 对于其他$4$种可能的请求顺序,假设$B$首先请求$x(x=2,3)$,可以有$A$请求到$1$,$B$请求到$x$。此时无论$x$是多少,$A$在请求到$x$之前不会释放$1$,$B$在请求到$1$之前不会释放$x$,所以可能会产生死锁。
\end{itemize}

\end{solution}
