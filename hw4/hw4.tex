\documentclass[11pt]{article}

\usepackage[a4paper]{geometry}
\geometry{left=2.0cm,right=2.0cm,top=2.5cm,bottom=2.5cm}

\usepackage{ctex} % 支持中文的LaTeX宏包
\usepackage{float} % 防止图片乱序
\usepackage{xeCJK} % 中文字体
\usepackage{amsmath,amsfonts,graphicx,subfigure,amssymb,bm,amsthm,mathrsfs,
            mathtools,breqn} % 数学公式和符号的宏包集合
\usepackage{algorithm,algorithmicx} % 算法和伪代码
\usepackage[noend]{algpseudocode} % 算法和伪代码
\usepackage{fancyhdr} % 自定义页眉页脚
\usepackage[framemethod=TikZ]{mdframed} % 创建带边框的框架
\usepackage{fontspec} % 字体设置
\usepackage{adjustbox} % 调整盒子大小
\usepackage{fontsize} % 设置字体大小
\usepackage{tikz,xcolor} % 绘制图形和使用颜色
\usepackage{multicol} % 多栏排版
\usepackage{multirow} % 表格中合并单元格
\usepackage{pdfpages} % 插入PDF文件
\usepackage{listings} % 在文档中插入源代码
\usepackage{lstautogobble} % 去掉源代码多余的空格
\usepackage{xcolor} % 支持彩色
\usepackage{caption} % 支持图片标题
\usepackage{wrapfig} % 文字绕排图片
\usepackage{bigstrut,multirow,rotating} % 支持在表格中使用特殊命令
\usepackage{booktabs} % 创建美观的表格
\usepackage{circuitikz} % 绘制电路图
\usepackage{zhnumber} % 中文序号(用于标题)
\usepackage{tabularx} % 表格折行
\usepackage{enumitem} % 枚举列表
\usetikzlibrary{circuits.ee.IEC} % 使用欧洲风格的电路符号
\usetikzlibrary{circuits.logic.US} % 使用美国风格的逻辑门
\usetikzlibrary{shapes.geometric, arrows.meta, positioning} % 使用流程图的库

% 设置字体
\definecolor{bluekeywords}{rgb}{0.13, 0.13, 1}
\definecolor{greencomments}{rgb}{0, 0.5, 0}
\definecolor{redstrings}{rgb}{0.9, 0, 0}
\definecolor{graynumbers}{rgb}{0.5, 0.5, 0.5}
\lstset{
    autogobble, % 自动移除代码前的空白(对齐代码块)
    columns=fixed, % 使得空格不可拉伸
    showspaces=false, % 不显示空格
    showtabs=false, % 不显示制表符
    breaklines=true, % 自动换行
    showstringspaces=false, % 在字符串中不特别显示空格
    breakatwhitespace=true, % 仅在空白处进行自动换行
    escapeinside={(*@}{@*)}, % 设置一个特殊的标记,允许在代码中插入LaTeX代码
    commentstyle=\color{greencomments}, % 注释的颜色设置为绿色
    keywordstyle=\color{bluekeywords}, % 关键字的颜色设置为蓝色
    stringstyle=\color{redstrings}, % 字符串的颜色设置为红色
    numberstyle=\color{graynumbers}, % 行号的颜色设置为灰色
    identifierstyle=\color{orange}, % 标识符颜色
    basicstyle=\small\ttfamily, % 基本字体样式:小号的等宽字体
    frame=single, % 单线边框
    framesep=12pt, % 边框与代码的间隔
    framerule=0.75pt, % 边框宽度
    xleftmargin=13pt, % 左边距
    xrightmargin=7pt, % 右边距
    tabsize=4, % 制表符宽度
    captionpos=t, % 标题位置在顶部
}
\DeclareCaptionFont{white}{\color{white}}
\DeclareCaptionFormat{listing}{\colorbox[cmyk]{0.43, 0.35, 0.35,0.01}
{\parbox{\textwidth}{\hspace{15pt}#1#2#3}}}
\captionsetup[lstlisting]{format=listing,labelfont=white,textfont=white,
                          singlelinecheck=false, margin=0pt,
                          font={bf,footnotesize}}

\usepackage{hyperref} % 目录,章节,超链接
\hypersetup{
    bookmarks=true,  % 生成书签
    bookmarksnumbered=true  % 书签带章节编号
}

% 轻松引用, 可以用\cref{}指令直接引用, 自动加前缀.
% 例: 图片label为fig:1
% \cref{fig:1} => Figure.1
% \ref{fig:1}  => 1
\usepackage[capitalize]{cleveref}
% \crefname{section}{Sec.}{Secs.}
\Crefname{section}{Section}{Sections}
\Crefname{table}{Table}{Tables}
\crefname{table}{Table.}{Tabs.}

\setCJKmainfont{Source Han Serif CN}
\setmonofont{JetBrains Mono NL}
\punctstyle{kaiming}
% 偏好的几个字体, 可以根据需要自行加入字体ttf文件并调用
% 注意要在自己系统安装字体, 以供调用

\renewcommand{\emph}[1]{\begin{kaishu}#1\end{kaishu}}

% 对 section 等环境的序号使用中文
\renewcommand \thesection{\zhnum{section}、}
\renewcommand \thesubsection{\arabic{subsection}}
\renewcommand \thesubsubsection{(\arabic{subsubsection})}
\setlength{\parindent}{0pt} % 取消首行缩进

%%%%%%%%%%%%%%%%%%%%%%%%%%%
%改这里可以修改实验报告表头的信息
\newcommand{\name}{蓝宇舟}
\newcommand{\studentNum}{2022K8009918005}
\newcommand{\class}{B0911010Y-02}
%%%%%%%%%%%%%%%%%%%%%%%%%%%

\begin{document}

% 标题
\begin{center}
  \LARGE \bf 操作系统第四次作业
\end{center}

\begin{center}
  \emph{姓名} \underline{\makebox[7em][c]{\name}}
  % 如果名字比较长, 可以修改box的长度"8em"为其他值
  \emph{学号} \underline{\makebox[12em][c]{\studentNum}}
  \emph{班级} \underline{\makebox[15em][c]{\class}}\\
\end{center}


% 环境说明
% \section{环境说明}

\begin{itemize}
    \item 操作系统:{\tt Arch Linux x86_64}
    \item 内核版本:{\tt Linux 6.10.7-zen1-1-zen}
    \item 编译器版本:{\tt gcc (GCC) 14.2.1 20240805}
\end{itemize}


% 作业
\section{作业}

\subsection{
    现有 $5$ 个作业要在⼀台计算机上依次执⾏,它们的运⾏时间分别是 $2$ , $6$ , $9$ , $11$ 和 $X$ 。
}

\subsubsection{
    该以何种顺序运⾏这 $5$ 个作业,从⽽可以获得最短的平均响应时间?
}

因为作业依次执行,所以采用最短时间优先算法(STCF):

\begin{enumerate}
    \item 将所有作业按运行时间从小到大排序。
    \item 按排序后的顺序依次执行作业。
\end{enumerate}

根据 $X$ 的不同取值,可以得到不同的运行顺序:

\begin{itemize}
    \item 当 $X \le 2$ 时,运行顺序为 $X$ , $2$ , $6$ , $9$ , $11$ 。
    \item 当 $2 < X \le 6$ 时,运行顺序为 $2$ , $X$ , $6$ , $9$ , $11$ 。
    \item 当 $6 < X \le 9$ 时,运行顺序为 $2$ , $6$ , $X$ , $9$ , $11$ 。
    \item 当 $9 < X \le 11$ 时,运行顺序为 $2$ , $6$ , $9$ , $X$ , $11$ 。
    \item 当 $11 < X$ 时,运行顺序为 $2$ , $6$ , $9$ , $11$ , $X$ 。
\end{itemize}

\subsubsection{
    如果要获得最短的平均周转时间,该以何种顺序运⾏这 $5$ 个作业?
}

同样采用最短时间优先算法(STCF),和上一题结论相同。

\subsection{
    现有 $5$ 个作业(作业 $A$ 、 $B$ 、 $C$ 、 $D$ 、 $E$ )要在⼀台计算机上执⾏。假设它们在同⼀时间被提交,同时它们的运⾏时间分别是 $10$ 、 $8$ 、 $4$ 、 $12$ 和 $15$ 分钟。当使⽤以下 CPU 调度算法运⾏这 $5$ 个作业时,请计算平均等待时间(注意:假设作业切换可以瞬时完成,即开销为 $0$ )。
}

\subsubsection{
    Round robin 算法(使⽤该算法时,每个作业分到的 CPU 时间⽚相等)
}

所有作业按用时从短到长排序: $C$ 、 $B$ 、 $A$ 、 $D$ 、 $E$ 。使用 RR 算法时,执行过程如下:

\begin{enumerate}
    \item 前 $20$ 分钟内, $C$ 执行结束, $ABCDE$ 的等待时间均为 $16$ 分钟。
    \item 第 $20$ ~ $36$ 分钟, $B$ 执行结束。 $ABDE$ 的等待时间均为 $28$ 分钟, $C$ 的等待时间为 $16$ 分钟。
    \item 第 $36$ ~ $42$ 分钟, $A$ 执行结束。 $ADE$ 的等待时间均为 $32$ 分钟, $B$ 的等待时间为 $28$ 分钟, $C$ 的等待时间为 $16$ 分钟。
    \item 第 $42$ ~ $46$ 分钟, $D$ 执行结束。 $DE$ 的等待时间均为 $34$ 分钟, $A$ 的等待时间为 $32$ 分钟, $B$ 的等待时间为 $28$ 分钟, $C$ 的等待时间为 $16$ 分钟。
    \item 第 $46$ ~ $49$ 分钟, $E$ 执行结束。 $DE$ 的等待时间均为 $34$ 分钟, $A$ 的等待时间为 $32$ 分钟, $B$ 的等待时间为 $28$ 分钟, $C$ 的等待时间为 $16$ 分钟。
\end{enumerate}

平均等待时间为:

$$
T_{avg} =\frac{34 + 34 + 32 + 28 + 16}{5} = 28.8 \text{ 分钟}
$$

\subsubsection{
    优先级调度算法(作业 $A-E$ 的优先级分别是: $2,5,1,3,4$ ,其中 $5$ 是最⾼优先级, $1$ 是最低优先级)
}

所有作业的执行顺序为: $B$ 、 $E$ 、 $D$ 、 $A$ 、 $C$ 。等待时间如下:

\begin{table}[H]
    \centering
    \begin{tabular}{|c|c|c|c|c|c|}
    \hline
    作业 & $A$ & $B$ & $C$ & $D$ & $E$ \\ \hline
    等待时间(分钟) & $35$  & $0$  & $45$  & $23$  & $8$  \\ \hline
    \end{tabular}
\end{table}

平均等待时间为:

$$
T_{avg} = \frac{0 + 8 + 23 + 35 + 45}{5} = 22.2 \text{ 分钟}
$$

\subsubsection{
    先到先服务算法(假设作业的达到顺序是 $A$ , $B$ , $C$ , $D$ , $E$ )
}

所有作业的执行顺序为: $A$ 、 $B$ 、 $C$ 、 $D$ 、 $E$ 。等待时间如下:

\begin{table}[H]
    \centering
    \begin{tabular}{|c|c|c|c|c|c|}
    \hline
    作业 & $A$ & $B$ & $C$ & $D$ & $E$ \\ \hline
    等待时间(分钟) & $0$  & $10$  & $18$  & $22$  & $34$  \\ \hline
    \end{tabular}
\end{table}

平均等待时间为:

$$
T_{avg} = \frac{0 + 10 + 18 + 22 + 34}{5} = 16.8 \text{ 分钟}
$$

\subsubsection{
    最短时间优先算法
}

所有作业按用时从短到长排序: $C$ 、 $B$ 、 $A$ 、 $D$ 、 $E$ 。等待时间如下:

\begin{table}[H]
    \centering
    \begin{tabular}{|c|c|c|c|c|c|}
    \hline
    作业 & $A$ & $B$ & $C$ & $D$ & $E$ \\ \hline
    等待时间(分钟) & $12$  & $4$  & $0$  & $22$  & $34$  \\ \hline
    \end{tabular}
\end{table}

平均等待时间为:

$$
T_{avg} = \frac{0 + 4 + 12 + 22 + 34}{5} = 14.4 \text{ 分钟}
$$


\subsection{
    现有⼀个实时计算机系统,该系统需要处理两个控制任务,每个任务每 $20$ 毫秒运⾏⼀次,每次运⾏占⽤ $5$ 毫秒的 CPU 时间。此外,该系统还要处理⼀个每秒 $24$ 帧的视频,其中,每帧需要 $20$ 毫秒的 CPU 时间进⾏处理。请分析这个实时系统是否可调度?请写出分析过程。如果某个⽤⼾想在该系统上处理⼀个每秒 $60$ 帧的新视频(每帧同样需要 $20$ 毫秒的 CPU 时间),请问能否在该系统上处理新视频?
}
两个控制任务的 CPU 利用率相同,为

$$
P_{c1} = P_{c2} = \frac{5\text{ms}}{20\text{ms}} = 25\%
$$

对于每秒 $24$ 帧的视频,CPU 利用率为:

$$
P_{v24} = 20\text{ms/frame} \times 24\text{frame/s} = 48\%
$$

对于每秒 $60$ 帧的视频,CPU 利用率为:

$$
P_{v60} = 20\text{ms/frame} \times 60\text{frame/s} = 120\%
$$

在前一种情况下,CPU 的总利用率为:

$$
P_1 = P_{c1} + P_{c2} + P_{v24} = 98\% \le 100\%
$$

故这个实时系统可以调度。

而对于后一种情况,CPU 的总利用率为:

$$
P_2 = P_{v60} = 120\% > 100\%
$$

故不能该系统上处理新视频。


\end{document}
