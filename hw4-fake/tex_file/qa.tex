\section{作业}

\subsection{
    现有 $5$ 个作业要在⼀台计算机上依次执⾏,它们的运⾏时间分别是 $2$ , $6$ , $9$ , $11$ 和 $X$ 。
}

\subsubsection{
    该以何种顺序运⾏这 $5$ 个作业,从⽽可以获得最短的平均响应时间?
}

因为作业依次执行,所以采用最短时间优先算法(STCF):

\begin{enumerate}
    \item 将所有作业按运行时间从小到大排序。
    \item 按排序后的顺序依次执行作业。
\end{enumerate}

根据 $X$ 的不同取值,可以得到不同的运行顺序:

\begin{itemize}
    \item 当 $X \le 2$ 时,运行顺序为 $X$ , $2$ , $6$ , $9$ , $11$ 。
    \item 当 $2 < X \le 6$ 时,运行顺序为 $2$ , $X$ , $6$ , $9$ , $11$ 。
    \item 当 $6 < X \le 9$ 时,运行顺序为 $2$ , $6$ , $X$ , $9$ , $11$ 。
    \item 当 $9 < X \le 11$ 时,运行顺序为 $2$ , $6$ , $9$ , $X$ , $11$ 。
    \item 当 $11 < X$ 时,运行顺序为 $2$ , $6$ , $9$ , $11$ , $X$ 。
\end{itemize}

\subsubsection{
    如果要获得最短的平均周转时间,该以何种顺序运⾏这 $5$ 个作业?
}

同样采用最短时间优先算法(STCF),和上一题结论相同。

\subsection{
    现有 $5$ 个作业(作业 $A$ 、 $B$ 、 $C$ 、 $D$ 、 $E$ )要在⼀台计算机上执⾏。假设它们在同⼀时间被提交,同时它们的运⾏时间分别是 $10$ 、 $8$ 、 $4$ 、 $12$ 和 $15$ 分钟。当使⽤以下 CPU 调度算法运⾏这 $5$ 个作业时,请计算平均等待时间(注意:假设作业切换可以瞬时完成,即开销为 $0$ )。
}

\subsubsection{
    Round robin 算法(使⽤该算法时,每个作业分到的 CPU 时间⽚相等)
}

所有作业按用时从短到长排序: $C$ 、 $B$ 、 $A$ 、 $D$ 、 $E$ 。使用 RR 算法时,执行过程如下:

\begin{enumerate}
    \item 前 $20$ 分钟内, $C$ 执行结束, $ABCDE$ 的等待时间均为 $16$ 分钟。
    \item 第 $20$ ~ $36$ 分钟, $B$ 执行结束。 $ABDE$ 的等待时间均为 $28$ 分钟, $C$ 的等待时间为 $16$ 分钟。
    \item 第 $36$ ~ $42$ 分钟, $A$ 执行结束。 $ADE$ 的等待时间均为 $32$ 分钟, $B$ 的等待时间为 $28$ 分钟, $C$ 的等待时间为 $16$ 分钟。
    \item 第 $42$ ~ $46$ 分钟, $D$ 执行结束。 $DE$ 的等待时间均为 $34$ 分钟, $A$ 的等待时间为 $32$ 分钟, $B$ 的等待时间为 $28$ 分钟, $C$ 的等待时间为 $16$ 分钟。
    \item 第 $46$ ~ $49$ 分钟, $E$ 执行结束。 $DE$ 的等待时间均为 $34$ 分钟, $A$ 的等待时间为 $32$ 分钟, $B$ 的等待时间为 $28$ 分钟, $C$ 的等待时间为 $16$ 分钟。
\end{enumerate}

平均等待时间为:

$$
T_{avg} =\frac{34 + 34 + 32 + 28 + 16}{5} = 28.8 \text{ 分钟}
$$

\subsubsection{
    优先级调度算法(作业 $A-E$ 的优先级分别是: $2,5,1,3,4$ ,其中 $5$ 是最⾼优先级, $1$ 是最低优先级)
}

所有作业的执行顺序为: $B$ 、 $E$ 、 $D$ 、 $A$ 、 $C$ 。等待时间如下:

\begin{table}[H]
    \centering
    \begin{tabular}{|c|c|c|c|c|c|}
    \hline
    作业 & $A$ & $B$ & $C$ & $D$ & $E$ \\ \hline
    等待时间(分钟) & $35$  & $0$  & $45$  & $23$  & $8$  \\ \hline
    \end{tabular}
\end{table}

平均等待时间为:

$$
T_{avg} = \frac{0 + 8 + 23 + 35 + 45}{5} = 22.2 \text{ 分钟}
$$

\subsubsection{
    先到先服务算法(假设作业的达到顺序是 $A$ , $B$ , $C$ , $D$ , $E$ )
}

所有作业的执行顺序为: $A$ 、 $B$ 、 $C$ 、 $D$ 、 $E$ 。等待时间如下:

\begin{table}[H]
    \centering
    \begin{tabular}{|c|c|c|c|c|c|}
    \hline
    作业 & $A$ & $B$ & $C$ & $D$ & $E$ \\ \hline
    等待时间(分钟) & $0$  & $10$  & $18$  & $22$  & $34$  \\ \hline
    \end{tabular}
\end{table}

平均等待时间为:

$$
T_{avg} = \frac{0 + 10 + 18 + 22 + 34}{5} = 16.8 \text{ 分钟}
$$

\subsubsection{
    最短时间优先算法
}

所有作业按用时从短到长排序: $C$ 、 $B$ 、 $A$ 、 $D$ 、 $E$ 。等待时间如下:

\begin{table}[H]
    \centering
    \begin{tabular}{|c|c|c|c|c|c|}
    \hline
    作业 & $A$ & $B$ & $C$ & $D$ & $E$ \\ \hline
    等待时间(分钟) & $12$  & $4$  & $0$  & $22$  & $34$  \\ \hline
    \end{tabular}
\end{table}

平均等待时间为:

$$
T_{avg} = \frac{0 + 4 + 12 + 22 + 34}{5} = 14.4 \text{ 分钟}
$$


\subsection{
    现有⼀个实时计算机系统,该系统需要处理两个控制任务,每个任务每 $20$ 毫秒运⾏⼀次,每次运⾏占⽤ $5$ 毫秒的 CPU 时间。此外,该系统还要处理⼀个每秒 $24$ 帧的视频,其中,每帧需要 $20$ 毫秒的 CPU 时间进⾏处理。请分析这个实时系统是否可调度?请写出分析过程。如果某个⽤⼾想在该系统上处理⼀个每秒 $60$ 帧的新视频(每帧同样需要 $20$ 毫秒的 CPU 时间),请问能否在该系统上处理新视频?
}
两个控制任务的 CPU 利用率相同,为

$$
P_{c1} = P_{c2} = \frac{5\text{ms}}{20\text{ms}} = 25\%
$$

对于每秒 $24$ 帧的视频,CPU 利用率为:

$$
P_{v24} = 20\text{ms/frame} \times 24\text{frame/s} = 48\%
$$

对于每秒 $60$ 帧的视频,CPU 利用率为:

$$
P_{v60} = 20\text{ms/frame} \times 60\text{frame/s} = 120\%
$$

在前一种情况下,CPU 的总利用率为:

$$
P_1 = P_{c1} + P_{c2} + P_{v24} = 98\% \le 100\%
$$

故这个实时系统可以调度。

而对于后一种情况,CPU 的总利用率为:

$$
P_2 = P_{v60} = 120\% > 100\%
$$

故不能该系统上处理新视频。
