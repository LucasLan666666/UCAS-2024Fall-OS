\usepackage[a4paper]{geometry}
\geometry{left=2.0cm,right=2.0cm,top=2.5cm,bottom=2.5cm}

\usepackage[scheme=plain]{ctex} % 支持中文的LaTeX宏包
\usepackage{float} % 防止图片乱序
\usepackage{xeCJK} % 中文字体
\usepackage{fontspec} % 字体设置
\usepackage{lastpage} % 获取总页数
\usepackage{amsmath,amsfonts,graphicx,subfigure,amssymb,bm,amsthm,mathrsfs,
mathtools,breqn} % 数学公式和符号的宏包集合
\usepackage{algorithm,algorithmicx} % 算法和伪代码
\usepackage[noend]{algpseudocode} % 算法和伪代码
\usepackage[framemethod=TikZ]{mdframed} % 创建带边框的框架
\usepackage{adjustbox} % 调整盒子大小
\usepackage{fontsize} % 设置字体大小
\usepackage{tikz,xcolor} % 绘制图形和使用颜色
\usepackage{multicol} % 多栏排版
\usepackage{multirow} % 表格中合并单元格
\usepackage{pdfpages} % 插入PDF文件
\usepackage{listings} % 在文档中插入源代码
\usepackage{lstautogobble} % 去掉源代码多余的空格
\usepackage{xcolor} % 支持彩色
\usepackage{caption} % 支持图片标题
\usepackage{wrapfig} % 文字绕排图片
\usepackage{bigstrut,multirow,rotating} % 支持在表格中使用特殊命令
\usepackage{booktabs} % 创建美观的表格
\usepackage{circuitikz} % 绘制电路图
\usepackage{zhnumber} % 中文序号(用于标题)
\usepackage{tabularx} % 表格折行
\usepackage{enumitem} % 枚举列表
\usepackage{xparse} % 支持更多的命令定义
% \usepackage{fancyhdr} % 页眉页脚
\usetikzlibrary{circuits.ee.IEC} % 使用欧洲风格的电路符号
\usetikzlibrary{circuits.logic.US} % 使用美国风格的逻辑门
\usetikzlibrary{shapes.geometric, arrows.meta, positioning} % 使用流程图的库

\usepackage{hyperref} % 目录,章节,超链接
\hypersetup{
    bookmarks=true,  % 生成书签
    bookmarksnumbered=true  % 书签带章节编号
}

% 轻松引用, 可以用\cref{}指令直接引用, 自动加前缀.
% 例: 图片label为fig:1
% \cref{fig:1} => Figure.1
% \ref{fig:1}  => 1
\usepackage[capitalize]{cleveref}
% \crefname{section}{Sec.}{Secs.}
\Crefname{section}{Section}{Sections}
\Crefname{table}{Table}{Tables}
\crefname{table}{Table.}{Tabs.}
