\question {
    假设一台计算机上运行的一个进程其地址空间有 8 个虚页(每个虚页大小为 4KB,页号为 1 至 8),
    操作系统给该进程分配了 4 个物理页框(每个页框大小为 4KB),该进程对地址空间中虚页的访问顺序
    为 1 2 4 5 6 3 7 3 7 8。假设分配给进程的 4 个物理页框初始为空,请计算:
}

\begin{parts}
    \part {
        如果操作系统采用 CLOCK 算法管理内存,那么该进程访存时会发生多少次 page fault?当进程
        访问完上述虚页后,物理页框中保存的是哪些虚页?
    }
    \part {
        如果操作系统采用 LRU 算法管理内存,请再次回答上一小问两个问题。请回答虚页保存情况时,
        写出 LRU 链的组成,标明 LRU 端和 MRU 端。
    }
\end{parts}

\begin{solution}

\begin{parts}

\part
8 次 page fault。物理页框中保存的是虚页 6、3、7、8。

\part
8 次 page fault。物理页框中保存的是虚页 6、3、7、8。

LRU 链的组成为 6、3、7、8,LRU 端为 6,MRU 端为 8。

\end{parts}

\end{solution}
