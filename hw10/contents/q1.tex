\question {
    现有一块磁盘,扇区大小为 512B,假设其平均寻道时间是 4ms,旋转速率是 15000 RPM
    (每分钟15000转),传输带宽是 200MB/s,请计算:
}

\begin{parts}
    \part {
        当某一用户程序分别从磁盘上的一个文件中读取 256B,2KB,1MB 的数据时,这三种情况下的
        有效带宽各是多少?(注意:用户程序发送的文件数据读写请求经过文件系统处理后,发往磁盘
        的最小请求粒度是 4KB。计算有效带宽时不考虑软件层的时间开销)
    }
    \part {
        如果希望该用户程序读写该磁盘的有效带宽达到 180MB/s,则该程序的读写粒度应为多大?
    }
\end{parts}

\begin{solution}
\begin{parts}

\part

平均寻道时间题目已经给出:
$$
T_{\text{seek}} = 4 \text{ms}
$$

因为旋转速率是$15000 \text{RPM}$,所以每秒转数是$15000 / 60 = 250 \text{RPS}$,
则每转的时间是$1 / 250 = 4 \text{ms}$,所以平均旋转时间是
$$
T_{\text{rotation}} = 4 \text{ms} / 2 = 2 \text{ms}
$$

因为最小请求粒度是$4 \text{KB}$,所以$256 \text{B}$和$2 \text{KB}$都会向磁盘发出
$4 \text{KB}$的请求,而$1 \text{MB}$会向磁盘发出$1 \text{MB}$的请求。因此,可以
计算出三种情况下传输时间:
$$
\begin{aligned}
    T_{\text{transfer\_1}} &= \frac{4 \text{KB}}{200 \text{MB/s}} = 20 \mu s \\
    T_{\text{transfer\_2}} &= \frac{4 \text{KB}}{200 \text{MB/s}} = 20 \mu s \\
    T_{\text{transfer\_3}} &= \frac{1 \text{MB}}{200 \text{MB/s}} = 5 \text{ms}
\end{aligned}
$$

所以,三种情况下的总时间分别是:
$$
\begin{aligned}
    T_{\text{I/O\_1}} &= 4 \text{ms} + 2 \text{ms} + 20 \mu s = 6.02 \text{ms} \\
    T_{\text{I/O\_2}} &= 4 \text{ms} + 2 \text{ms} + 20 \mu s = 6.02 \text{ms} \\
    T_{\text{I/O\_3}} &= 4 \text{ms} + 2 \text{ms} + 5 \text{ms} = 11 \text{ms}
\end{aligned}
$$

可以得出三种情况下的有效带宽:
$$
\begin{aligned}
    R_{\text{I/O\_1}} &= \frac{256 \text{B}}{6.02 \text{ms}} = 0.0426 \text{MB/s} \\
    R_{\text{I/O\_2}} &= \frac{2 \text{KB}}{6.02 \text{ms}} = 0.335 \text{MB/s} \\
    R_{\text{I/O\_3}} &= \frac{1 \text{MB}}{11 \text{ms}} = 90.9 \text{MB/s}
\end{aligned}
$$


\part
由上一小问的计算结果可知,当用户程序的读写粒度为$1 \text{MB}$时,其有效带宽为
$90.9 \text{MB/s}$。所以读写粒度应该不小于$1 \text{MB}$。设读写粒度为$X \text{MB}$,
则有:
$$
    R_{\text{I/O}} = \frac{X \text{MB}}{T_{\text{seek}} + T_{\text{rotation}} + \frac{X \text{MB}}{200 \text{MB/s}}}
$$

代入$R_{\text{I/O}} = 180 \text{MB/s}$,解方程得:
$$
X = 10.8 \text{MB}
$$

所以,用户程序的读写粒度应该不小于$10.8 \text{MB}$。

\end{parts}
\end{solution}
