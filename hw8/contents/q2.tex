\question {
    假设一台计算机使用 32-bit 的虚拟地址空间和三级页表,虚地址的划分为 8-bit | 6-bit | 6-bit | 12-bit(注:8 bit 对应为第一级页表的地址,以此类推),请计算:
}

\begin{parts}
    \part 该计算机系统的页大小是多少?
    \part 该三级页表一共能索引多少个页?
    \part 现有一个程序的代码段大小为 128KB,数据段为 66KB,栈大小为 8KB,则在使用上述三级
    页表时,最少需要占用多少个物理页框?最多会占用多少个物理页框?(注:假设程序各段在地址空间
    中的布局可以自行决定)
    \part 假设该计算机使用一级页表进行地址空间管理,则上一问中程序需要占用多少个物理页框?
\end{parts}


\begin{solution}

\begin{parts}

\part
页大小为
$$
    2^{12} = 4096 \text{ B} = 4 \text{ KB}
$$

\part
三级页表一共能索引的页数为
$$
    2^{8+6+6} = 2^{20} \text{ Pages} = 1 \text{ M Pages}
$$

\part {
    最少情况:代码段、数据段和栈分别占据$31$、$17$和$2$页,即共占据$51$页,最少需要占用$51$个物理
    页框,少于一级页表能索引的页数$2^6=64$。因此可以放在同一个一级页表内。所以最少需要的物理页框数
    为$1$个三级页表,$1$个二级页表,$1$个一级页表,以及虚拟页本身的$51$个物理页框,共$54$个物理页框。

    最多情况:代码段、数据段和栈在虚拟地址空间分散排布,由于这三者最少需要两页,所以对于其中任何一个,
    都可以跨越一级页表和二级页表。因此最多需要的物理页框数为$1$个三级页表,$2 \times 3 = 6$个二级
    页表,$2 \times 3 = 6$个一级页表,以及虚拟页本身的$51$个物理页框,共$64$个物理页框。
}

\parts
如果使用一级页表进行地址空间管理,则只需要$1$个一级页表和虚拟页本身的$51$个物理页框,共$52$个物理页框。

\end{parts}

\end{solution}
